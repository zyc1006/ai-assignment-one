\documentclass{article}
\begin{document}
\title{AI - Assignment One: The Plonk Planning Problem}
\date{2012-10-08}
\author{
	Zhou, Yucheng\\
	\texttt{Yucheng.Zhou.3489@student.uu.se}
	\and
	Siegmund, Martin\\
	\texttt{Marting.Siegmund.8845@student.uu.se}
	\and
	Olsson, Joakim\\
	\texttt{Joakim.Olsson.0581@student.uu.se}
	\and
	Gustafsson, Daniel\\
	\texttt{Daniel.Gustafsson.6481@student.uu.se}
}
\maketitle 

\section*{Introduction}
\" Servicing Plonks has been a crucial task for humankind, even since before the taming of fire. Other species like the Neanderthals went extinct solely because of their lacking skills in this endeavour. This is why every computer scientist can be very proud to be a part of a course that makes this important undertaking easier. \"

\section*{Algorithm}
When we started to discuss what algorithm to use, we first decided to use A*. But as we then reasoned, you could randomly do actions and still reach 1000 serviced plonks. Since we can also make an infinite ammount of spunks and can find an infinite ammount of plonks, we can never reach a point where we run out of resources. When we realised this we changed to a greedy algorithm since the backtracking of A* wasn't necessary.
\\ \\
Our algorithm first checks that none of the input is 0 since that could lead to an infinite loop. After this first check we do the following:
\begin{enumerate}
	\item Check if we've reached the goal of 1000 serviced plonks yet. If we have reached this, stop, otherwise continue.
	\item Recover resources. Since resources are used during the hour, here the resources that were used the previous hour are made available and any bligs coming out of service are made available with 8 hours of working time.
	\item Check what of the 5 possible actions we can do and do them, we choose to prioritize the actions like this, with the most prioritized one first:
	\begin{enumerate}
		\item Service as many plonks as possible.
		\item With the remaining resources, find as many plonks as possible.
		\item With the remaining resources, service broken bligs using the fast service.
		\item With the remaining resources, service broken bligs using the slow service.
		\item With the remaining resources, make more spunks.
	\end{enumerate}
	\item Increase time.
	\item Repeat
\end{enumerate}

This priority of actions was decided upon since the goal is to service plonks. To service plonks they need to be found, and we also need working bligs. Furthermore Finding Plonks and Servicing Bligs don't have any employees or tools they both need, so unless we are short of Spunks the order does not matter here. And to make everything we need spunks. 
\\ \\
This ordering also prevents infinite loops since the producing actions (making spunks and finding plonks) are prioritized less then the respective consuming actions. Making spunks is prioritized lowest and finding plonks is prioritized less then servicing plonks.

\section*{Servicing bligs}
We made the decision early on to only service broken bligs. During development of the algorithm we chose to prioritize the fast service higher then the slower service because the fast service is fast.

\section*{Suggestions}
To give the user suggestions as what to change concerning the resources, we calculate the average unused employees and tools. We then consider to take a ressource with the highest unused time and give one to the lowest. The program takes another run with changed values afterwards to check if the change really improves the time and then gives the user the suggestion.
This way, we keep unused ressources to a minimum and still check if we didn't just find a weird situation.

\section*{Heuristic function}
We calculated an optimistic function for some starting states and the hours weren't that much lower than our numbers, so they are pretty good.

\section*{Conclusions}
When we started solving this with A* we made an optimistic Heuristic function. This function estimated the time to reach 1000 Serviced Plonks to 25 hours for the resource distribution (5 Spunkees, 6 Bligees, 12 Plinks, 16 Workbenches and 11 Bligs) and our greedy algorithm arrived at 1000 Serviced Plonks at 28 hours.
\\ \\
The good solution together with our quick runtime, less then 1 second, allows us to conclude that our algorithm is sufficiently good.

\end{document}
